% jsonresume-doc.tex
% Documentation for the jsonresume LaTeX package
% Compile with: lualatex jsonresume-doc.tex

\documentclass[11pt,a4paper]{article}

% Page geometry
\usepackage[
    top=1in,
    bottom=1in,
    left=1in,
    right=1in
]{geometry}

% Required packages
\usepackage{hyperref}
\usepackage{xcolor}
\usepackage{listings}
\usepackage{longtable}
\usepackage{booktabs}
\usepackage{parskip}
\usepackage{fancyvrb}
\usepackage{microtype}

% Package version info (updated by build script)
\def\packageversion{1.0.1}
\def\packagedate{2026/01/31}

% Hyperlink styling
\hypersetup{
    colorlinks=true,
    linkcolor=blue!70!black,
    urlcolor=blue!70!black,
    citecolor=blue!70!black,
    pdftitle={jsonresume Package Documentation},
    pdfauthor={Lukas Wolfsteiner},
    pdfsubject={LaTeX package for rendering JSON Resume data},
    pdfkeywords={LaTeX, LuaLaTeX, JSON Resume, curriculum vitae, resume}
}

% Code listing style
\lstset{
    basicstyle=\ttfamily\small,
    keywordstyle=\color{blue!70!black}\bfseries,
    commentstyle=\color{green!50!black}\itshape,
    stringstyle=\color{red!70!black},
    backgroundcolor=\color{gray!10},
    frame=single,
    framerule=0.5pt,
    rulecolor=\color{gray!50},
    breaklines=true,
    breakatwhitespace=true,
    showstringspaces=false,
    tabsize=2,
    xleftmargin=1em,
    xrightmargin=1em,
    aboveskip=1em,
    belowskip=1em
}

% Custom commands for documentation
\newcommand{\pkg}[1]{\textsf{#1}}
\newcommand{\cmd}[1]{\texttt{\textbackslash#1}}
\newcommand{\meta}[1]{\textit{$\langle$#1$\rangle$}}
\newcommand{\marg}[1]{\texttt{\{}\meta{#1}\texttt{\}}}
\newcommand{\oarg}[1]{\texttt{[}\meta{#1}\texttt{]}}

\title{%
    \pkg{jsonresume} Package Documentation\\[0.5em]
    \large A LuaLaTeX Package for Rendering JSON Resume Data\\[0.5em]
    \normalsize Version \packageversion\ (\packagedate)
}
\author{Lukas Wolfsteiner\thanks{\url{mailto:lukas@wolfsteiner.media}}}
\date{}

\begin{document}

\maketitle

\begin{abstract}
The \pkg{jsonresume} package provides a simple way to render resume data from
JSON Resume format files into professionally formatted LaTeX documents. It supports
loading data from local files or remote URLs, includes optional schema validation,
and provides commands for rendering individual sections or the complete resume.
This package requires LuaLaTeX.
\end{abstract}

\tableofcontents

\newpage

%-------------------------------------------------------------------------------
\section{Introduction}
%-------------------------------------------------------------------------------

The \pkg{jsonresume} package allows you to separate your resume content from its
presentation. By storing your resume data in the standardized JSON Resume format
(\url{https://jsonresume.org/}), you can:

\begin{itemize}
    \item Maintain a single source of truth for your resume data
    \item Render your resume in different formats and styles
    \item Load resume data from local files or web URLs
    \item Validate your data against the JSON Resume schema
\end{itemize}

The package supports all 12 sections defined in the JSON Resume schema:
basics, work, volunteer, education, awards, certificates, publications,
skills, languages, interests, references, and projects.

%-------------------------------------------------------------------------------
\section{Requirements}
%-------------------------------------------------------------------------------

\begin{itemize}
    \item \textbf{LuaLaTeX} -- This package requires LuaLaTeX and will not work
          with pdfLaTeX or XeLaTeX.
    \item \textbf{curl} -- Required for loading resume data from URLs. This is
          pre-installed on most systems.
    \item \textbf{Shell escape} -- Required for URL loading. Compile with
          \texttt{--shell-escape} option.
\end{itemize}

%-------------------------------------------------------------------------------
\section{Installation}
%-------------------------------------------------------------------------------

\subsection{From CTAN / TeX Live}

If the package is installed via your TeX distribution (TeX Live, MiKTeX), it
should be available automatically. Simply use:

\begin{lstlisting}[language={[LaTeX]TeX}]
\usepackage{jsonresume}
\end{lstlisting}

\subsection{Manual Installation}

Copy \texttt{jsonresume.sty} and \texttt{jsonresume.lua} to either:

\begin{enumerate}
    \item Your project directory (alongside your \texttt{.tex} file), or
    \item Your local texmf tree:
\begin{lstlisting}[language=bash]
mkdir -p ~/texmf/tex/latex/jsonresume
cp jsonresume.sty jsonresume.lua ~/texmf/tex/latex/jsonresume/
\end{lstlisting}
\end{enumerate}

%-------------------------------------------------------------------------------
\section{Quick Start}
%-------------------------------------------------------------------------------

Here is a minimal example to get started:

\begin{lstlisting}[language={[LaTeX]TeX}]
\documentclass[11pt,letterpaper]{article}
\usepackage[margin=0.5in]{geometry}
\usepackage{jsonresume}

\begin{document}

% Load resume data from a file
\resumefromfile{resume.json}

% Render the complete resume
\renderresume

\end{document}
\end{lstlisting}

Compile with:
\begin{lstlisting}[language=bash]
lualatex --shell-escape yourfile.tex
\end{lstlisting}

%-------------------------------------------------------------------------------
\section{Package Options}
%-------------------------------------------------------------------------------

\subsection{Strict Mode}

Enable strict mode to receive warnings about schema violations:

\begin{lstlisting}[language={[LaTeX]TeX}]
\usepackage[strict]{jsonresume}
\end{lstlisting}

In strict mode, the package warns about:
\begin{itemize}
    \item Unknown top-level sections
    \item Missing required fields (e.g., \texttt{basics.name})
    \item Invalid date formats (should be YYYY, YYYY-MM, or YYYY-MM-DD)
    \item Invalid URL formats
    \item Type mismatches (e.g., string instead of array)
\end{itemize}

Warnings appear in the LaTeX log file.

%-------------------------------------------------------------------------------
\section{Commands}
%-------------------------------------------------------------------------------

\subsection{Loading Data}

\begin{longtable}{@{}lp{8cm}@{}}
\toprule
\textbf{Command} & \textbf{Description} \\
\midrule
\endhead
\cmd{resumefromfile}\marg{path} & Load resume data from a local JSON file \\
\cmd{resumefromurl}\marg{url} & Load resume data from a remote URL (requires shell escape) \\
\bottomrule
\end{longtable}

\subsection{Rendering Sections}

\begin{longtable}{@{}lp{8cm}@{}}
\toprule
\textbf{Command} & \textbf{Description} \\
\midrule
\endhead
\cmd{renderresume} & Render all available sections in standard order \\
\cmd{renderresumecore} & Render core sections only (basics, work, education, skills) \\
\cmd{resumebasics} & Render name, contact information, and summary \\
\cmd{resumework}\oarg{title} & Render work experience section \\
\cmd{resumevolunteer}\oarg{title} & Render volunteer experience section \\
\cmd{resumeeducation}\oarg{title} & Render education section \\
\cmd{resumeawards}\oarg{title} & Render awards and honors section \\
\cmd{resumecertificates}\oarg{title} & Render certifications section \\
\cmd{resumepublications}\oarg{title} & Render publications section \\
\cmd{resumeskills}\oarg{title} & Render skills section \\
\cmd{resumelanguages}\oarg{title} & Render languages section \\
\cmd{resumeinterests}\oarg{title} & Render interests section \\
\cmd{resumereferences}\oarg{title} & Render references section \\
\cmd{resumeprojects}\oarg{title} & Render projects section \\
\bottomrule
\end{longtable}

The optional \meta{title} argument allows you to customize the section heading.
For example, \verb|\resumework[Professional Experience]| will use
``Professional Experience'' instead of the default ``Experience''.

\subsection{Validation Commands}

\begin{longtable}{@{}lp{8cm}@{}}
\toprule
\textbf{Command} & \textbf{Description} \\
\midrule
\endhead
\cmd{resumevalidate} & Manually trigger validation \\
\cmd{resumevalidationsummary} & Print validation summary to document \\
\cmd{resumewarningcount} & Output the number of validation warnings \\
\bottomrule
\end{longtable}

\subsection{Data Access (Advanced)}

For advanced users who want to access individual data fields:

\begin{longtable}{@{}lp{8cm}@{}}
\toprule
\textbf{Command} & \textbf{Description} \\
\midrule
\endhead
\cmd{jrget}\marg{path} & Get a value using dot notation \\
\cmd{jrifexists}\marg{path}\marg{yes}\marg{no} & Conditional based on path existence \\
\bottomrule
\end{longtable}

Example:
\begin{lstlisting}[language={[LaTeX]TeX}]
Name: \jrget{basics.name}
Email: \jrget{basics.email}

\jrifexists{basics.phone}{Phone: \jrget{basics.phone}}{}
\end{lstlisting}

%-------------------------------------------------------------------------------
\section{JSON Resume Format}
%-------------------------------------------------------------------------------

The package supports the full JSON Resume schema. Here is an abbreviated example
showing the structure:

\begin{lstlisting}
{
  "basics": {
    "name": "Jane Doe",
    "label": "Software Engineer",
    "email": "jane@example.com",
    "phone": "+1-555-123-4567",
    "url": "https://janedoe.dev",
    "summary": "Experienced software engineer...",
    "location": {
      "city": "San Francisco",
      "region": "CA",
      "countryCode": "US"
    },
    "profiles": [
      {
        "network": "GitHub",
        "username": "janedoe",
        "url": "https://github.com/janedoe"
      }
    ]
  },
  "work": [
    {
      "name": "Company Name",
      "position": "Software Engineer",
      "startDate": "2020-01",
      "endDate": "2023-12",
      "highlights": ["Built features", "Led teams"]
    }
  ],
  "education": [...],
  "skills": [...],
  "projects": [...]
}
\end{lstlisting}

For the complete schema specification, see \url{https://jsonresume.org/schema/}.

%-------------------------------------------------------------------------------
\section{Custom Layout Example}
%-------------------------------------------------------------------------------

You can customize the order and titles of sections:

\begin{lstlisting}[language={[LaTeX]TeX}]
\documentclass[11pt]{article}
\usepackage[margin=0.75in]{geometry}
\usepackage[strict]{jsonresume}

\begin{document}
\resumefromfile{resume.json}

% Custom order and custom titles
\resumebasics
\resumeskills[Technical Expertise]
\resumework[Professional Experience]
\resumeprojects[Side Projects]
\resumeeducation[Academic Background]
\resumecertificates[Professional Certifications]
\resumelanguages

\end{document}
\end{lstlisting}

%-------------------------------------------------------------------------------
\section{Troubleshooting}
%-------------------------------------------------------------------------------

\subsection{``This package requires LuaLaTeX''}

The package only works with LuaLaTeX. Make sure you compile with:
\begin{lstlisting}[language=bash]
lualatex --shell-escape yourfile.tex
\end{lstlisting}

\subsection{URL Loading Fails}

If loading from URLs doesn't work:
\begin{enumerate}
    \item Ensure you're using the \texttt{--shell-escape} option
    \item Verify that \texttt{curl} is installed and accessible
    \item Check that the URL is accessible from your machine
\end{enumerate}

\subsection{Validation Warnings}

If you see validation warnings in strict mode, check the LaTeX log file for details.
Common issues include:
\begin{itemize}
    \item Date formats should be YYYY, YYYY-MM, or YYYY-MM-DD
    \item URLs should include the protocol (https://)
    \item Arrays (like \texttt{highlights}) should be JSON arrays, not strings
\end{itemize}

%-------------------------------------------------------------------------------
\section{License}
%-------------------------------------------------------------------------------

Copyright \copyright\ 2026 Lukas Wolfsteiner

This package is released under the MIT License. You are free to use, modify,
and distribute this package, provided that the copyright notice and license
terms are preserved.

See the LICENSE file included with this package for the full license text.

%-------------------------------------------------------------------------------
\section{Bug Reports and Contributions}
%-------------------------------------------------------------------------------

Bug reports, feature requests, and contributions are welcome at:

\url{https://github.com/dotWee/TeXed-JSON-Resume}

\end{document}
